\documentclass[11pt]{article}

\usepackage[style=prsb,natbib=true,uniquename=false,uniquelist=false,sortcites=false]{biblatex}
\addbibresource{refs.bib}

\usepackage{graphicx}
\usepackage{xcolor}
\usepackage{booktabs}
\usepackage{array}
\usepackage[letterpaper]{geometry}
\geometry{hmargin={1in,1in},vmargin={0.7in,0.7in}}

\usepackage[hidelinks]{hyperref}
\usepackage[inline]{enumitem}
\newlist{ilnum}{enumerate*}{1}
\setlist[ilnum]{ label=(\Roman*) }

\usepackage{amsmath}
\usepackage{amsthm,amssymb}
\usepackage{mleftright}
\mleftright

%%%%%%%%%%%%%%%%%%%%%%%%%%%%%%%%%%%%%%%%%%%%%%%%%%%%%%%%%%%%%%%%%
% XeLaTeX font settings
\usepackage{fontspec}
\defaultfontfeatures{Ligatures=TeX}
\usepackage{unicode-math}

\setmainfont{Minion Pro}
\setsansfont[Scale=MatchLowercase]{Myriad Pro}
\setmonofont[Scale=MatchLowercase]{Fira Code}
\setmathfont[Scale=MatchLowercase]{TeX Gyre Pagella Math}
\setmathfont[range=up/{num,latin,Latin,greek,Greek}]{Minion Pro}
\setmathfont[range=it/{latin,Latin,greek,Greek}]{Minion Pro It}
\setmathfont[range=bfup/{latin,Latin,greek,Greek}]{Minion Pro Bold}
\setmathfont[range=bfit/{latin,Latin,greek,Greek}]{Minion Pro Bold It}
%%%%%%%%%%%%%%%%%%%%%%%%%%%%%%%%%%%%%%%%%%%%%%%%%%%%%%%%%%%%%%%%%

\usepackage[explicit]{titlesec}
\titleformat{\section}        {\large \bfseries \scshape}{\thesection.}{1ex}{#1}
\titleformat{\subsection}     {\bfseries \itshape}{\thesubsection.}{1ex}{#1}
\titleformat{\subsubsection}  [runin]{\bfseries}{\thesubsubsection.}{1ex}{#1}

\titlespacing{\section}       {0pt}{1.5ex plus 0.5ex minus 0.5ex}{0.5ex plus 0.5ex minus 0.5ex}
\titlespacing{\subsection}    {0pt}{1.5ex plus 0.5ex minus 0.5ex}{0.0ex plus 0.25ex minus 0.25ex}
\titlespacing{\subsubsection} {0pt}{1.5ex plus 0.5ex minus 0.5ex}{1em}

\usepackage{setspace}
\usepackage{titling}
\usepackage{xparse}
\usepackage{lineno}

%%% new commands and macros
\newcommand{\der}{\mathop{}\!\mathrm{d}}
\newcommand{\Oh}[1]{O\left(#1\right)}
\newcommand{\ExpectationOperator}{\mathrm{E}}
\NewDocumentCommand \E {d[] d()} {%
  \IfNoValueTF {#1} {%
    \IfNoValueTF {#2} {%
      \mathop{\kern0pt\ExpectationOperator}\nolimits%
    }{%
      \mathop{\kern0pt\ExpectationOperator}\nolimits\left(#2\right)%
    }}{%
    \IfNoValueTF {#2} {%
      \mathop{\kern0pt\ExpectationOperator#1}\nolimits%
    }{%
      \mathop{\kern0pt\ExpectationOperator#1}\nolimits\left(#2\right)%
    }%
  }
}
\DeclareMathOperator{\Var}{Var}
\DeclareMathOperator{\Cov}{Cov}
\newcommand{\Fst}{F_{\mathrm{ST}}}
\newcommand{\mean}[1]{\overline{#1}}
\newcommand{\w}{w}
\newcommand{\ess}[1]{#1^*}

\begin{document}

\begin{titlingpage}
\setlength{\droptitle}{2em}
\pretitle{\begin{center}\LARGE}
\posttitle{\par\end{center}\vskip 2.5em}

\title{\scshape Evolutionary dynamics, equilibrium selection, and what population and quantitative genetics have taught evolutionary game theory}
\author{Jeremy Van Cleve}
\date{}
\maketitle

\vfill

\noindent
Department of Biology\\
University of Kentucky\\
Lexington, KY 40506 USA\\[1em]
phone: 859-218-3020\\
fax: 859-257-1717\\[1em]
e-mail: \href{mailto:jvancleve@uky.edu}{jvancleve@uky.edu}

\vspace{2em}

\begin{flushright} \textit{Date modified: \today} \end{flushright}
\end{titlingpage}

\linenumbers
\onehalfspacing
\begin{abstract}

The application of game theory to animal behavior by Maynard Smith and Price invigorated evolutionary biology by suggesting that the complex dynamics among natural selection, genes, and behavior could be analyzed using an evolutionary refinement of the Nash equilibrium, namely the evolutionarily stable strategy (ESS)\@. As implied by ``equilibrium'' and ``stable'', an ESS is a predicted long-term outcome that an evolutionary process may reach via natural selection and mutation. ESS analysis has been particularly important in highlighting the importance of social dilemmas in biology where the evolution of cooperative behavior can be undermined by the individual fitness costs of that behavior. However, even simple games like the stag-hunt or coordination game can have multiple ESSs, which begs the question, which ESS is more likely to evolve? Unfortunately, ESS analysis does not reveal which ESS is more likely to evolve nor does it even show how genetic or demographic factors affect the speed of approach to an ESS. Progress on determining which ESSs are more ``realistic'' has been made using tools from stochastic processes and population and quantitative genetics including stochastic stability, absorption times, and G-matrix theory. In this paper, we will review the use of these tools and provide novel insight into how they can predict social evolution as a function of genetic and demographic processes.

\end{abstract}

\vspace{3em}
\noindent {\bfseries Key words}\\

\newpage

\section{Outline/Brainstorm}

\begin{enumerate}
  \item Intro
        \begin{enumerate}
          \item Little intro about Maynard Smith and Price
          \item Game theory as general optimization and search for fixed points
          \item Connect back to Hamilton 1964 and goal of optimization with relatedness
          \item Evolutionary process is dynamic but still has ``fixed points''
          \item Evolutionary change can be decomposed and each piece affects location and stability of fixed points
          \item Evol Game theory captures selection but not demography and genetic forces, which can have strong effects.
        \end{enumerate}
  \item Ok, so thinking more about this, the two main pieces in certain ways are the fitness or success measure and whether maximizing it leads a correct predictions. pop mean fitness as measure fails due to genetics and due to frequency dependence. hamilton suggested inclusive fitness. later authors showed that this is equivalent to an invasion fitness perspective. maximization of this leads to ESS. Stochasticity though due to mutation/drift/etc leads to stationary distribution...
  \item Extra things
        \begin{enumerate}
          \item emphasize the incompleteness of the review. Maybe cite a few ``relevant'' Perc papers?
        \end{enumerate}
\end{enumerate}

%%
the flow for the intro should be something like: maynard smith setup this false dichotomy: ESS is about ``individual'' level selection when really there are two issues, ESS as tool of optimality and then the application of that tool to kin/group models. this requires explaining how ESS is a tool of optimality and how a particular optimality measure, namely lineage growth rate, accounts for kin and group selection since it can captures selection at all levels from the gene on up. accounting for other evolutionary forces is more difficult but one way forward uses the fixation probability and allows mutation/demography etc to affect fixation probabilities and hence transition rates.


\clearpage
\section{Introduction}

Although Richard Lewontin was the first to introduce game theory into biology in 1961 \cite{Lewontin:1961}, very few papers on the topic of game theory and biology were published in the following decade. It wasn't until 1973 when John Maynard Smith and George Price introduced the evolutionarily stable strategy (ESS) and applied it to the study of animal behavior \cite{Maynard-Smith:Price:1973} that biologists more widely came to appreciate the relevance and utility of game theoretic concepts and tools for questions in evolution biology and ecology. Specifically, Maynard Smith and Price posited that individual fitness could be viewed as an analog of the game-theoretic notion of ``utility'', which is the quantity that measures what agents optimize in pursuit of their objectives \cite{Myerson:1991}. Viewed this way, the ESS is a refinement of the famous Nash equilibrium \cite{Nash:1950}. Maynard Smith explained the ESS concept in more detail in an important paper in 1974 \cite{Maynard-Smith:1974}, introduced the famous ``Hawk-Dove'' game in his study of asymmetric games in 1976 \cite{Maynard-Smith:Parker:1976}, and summarized the nascent field of evolutionary game theory in his now classic 1982 book \cite{MaynardSmith:1982}. In the 50 years since the 1973 paper, evolutionary game theory has become an essential tool in evolutionary and behavioral ecology with rich theoretical work that delves into foundational evolutionary and game-theoretic concepts and mathematical and simulation models that predict ESSs for almost innumerable biological systems. Evolutionary game theory has also strongly influenced the social sciences including, naturally, economics, political science, psychology, and anthropology and has drawn applied mathematicians, computer scientists and physicists into mathematical biology.

The role of evolutionary game theory and the ESS method specifically in evolutionary biology has been at the crux of some of the most important conceptual debates in the field including the role of natural selection and adaptation vis-a-vis other forces \cite[e.g.,][]{MaynardSmith:1978,Gould:Lewontin:1979,Lewontin:1979,Orzack:Sober:1994,Gardner:2017,Kern:Hahn:2018,Jensen:Payseur:2019} and the importance of kin and group selection relative to individual selection
\cite[e.g.,][]{Maynard-Smith:1964,Hamilton:1963,Price:1972:cov,Wilson:Wilson:2007,Leigh:2010,Akcay:Van-Cleve:2012,West:Griffin:2007,Gardner:Grafen:2009,Nowak:Tarnita:2010,Abbot:Abe:2011,Allen:Nowak:2013,Birch:2014,Birch:2017,Nowak:McAvoy:2017}. These debates might be captured in part by the following two questions: \begin{ilnum} \item \label{q:I} How do ESS models that focus on individual fitness capture the effects of kin selection or group selection? \item \label{q:II} How does an ESS account for mutation, genetic drift, and other evolutionary forces? \end{ilnum} In this work, I will describe how conceptual advances in evolutionary theory since Maynard Smith and Price \cite{Maynard-Smith:Price:1973} have shed light on both of these questions and have led to a broader understanding of the ESS method and a more integrative framework for understanding how multiple evolutionary forces in complex populations can lead to diverse phenotypes.

Question \ref{q:I} derives in part from how Maynard Smith introduced his work on ESSs; he argued that an ESS could provide an explanation for the evolution of behaviors where ``selection acts entirely at the individual level, but in which the success of any particular strategy depends on what strategies are adopted by other members of the population'' \cite[][p. 210]{Maynard-Smith:1974} and is not ``due to group or species selection'' \cite[p. 15]{Maynard-Smith:Price:1973} or selection on ``close relatives'' \cite[p. 210]{Maynard-Smith:1974} as might be the case for kin selection \cite{Hamilton:1964}. In setting up this dichotomy, Maynard Smith implies that the ESS method may not apply when kin or group selection are involved and that kin or group selection models might lead to different results when applied to the same biological scenario. As I will describe below, evolutionary theorists have since realized (including Maynard Smith himself \cite[p. 33]{MaynardSmith:1978}) that the ESS method is really orthogonal to issues of individual vs. group vs. kin selection and instead captures in what sense evolution via natural selection optimizes fitness given a specific measure of fitness. Issues of individual vs. group vs. kin selection are about the ``units'' or ``levels'' at which selection acts and how fitness should be measured to account for selection at those levels. Lewontin also hints at the levels of selection question in his 1961 game theory paper when he discusses the relative merits of individual-level measures of fitness like intrinsic or Malthusian growth rate $r$ versus population-level measures like mean fitness $\mean{w}$ \cite[p. 400-401]{Lewontin:1961} as analogs of utility. The levels of selection question became a major topic of study in evolutionary theory \cite[e.g.,][]{Lewontin:1970,Dawkins:1982,Wilson:Sober:1989,Maynard-Smith:Szathmary:1995,Wilson:1997,Michod:1999,Michod:2006,Szathmary:2015} and philosophy of biology \cite{Hull:1980,Brandon:1982,Damuth:Heisler:1988,Lloyd:1992,Lloyd:1994,Sober:Wilson:1994,Okasha:2006,Okasha:2016} and continues to generate substantial research \cite[e.g.,][]{Black:Bourrat:2020,Cooney:Mori:2022,Veit:2022}. For the purpose of explaining how an ESS can capture selection at multiple levels and among relatives or kin, I will argue below that the right measure of utility is the ``lineage fitness'' \cite{Lehmann:Alger:2015,Akcay:Van-Cleve:2016,Lehmann:Mullon:2016} of a mutant allele at a single genetic locus, which can be shown to be functionally equivalent to a measure of inclusive fitness \cite{Lehmann:Mullon:2016,Lehmann:Rousset:2020}.

The origin of question \ref{q:II} rests in a different set of debates in evolutionary biology regarding the relative role of natural selection vis-a-vis other evolutionary forces, such as mutation, recombination, and gene flow, in explaining organismal phenotypes. Early in the 20th century, R. A. Fisher's ``fundamental theorem of natural selection'' (FTNS) \cite{Fisher:1930} established a mathematical expression of the importance of natural selection. The FTNS states that the increase in the mean fitness of a population is equal to the genetic variance in fitness and thus seems to imply that populations always become better adapted to their environments and even that fitness is maximized over the long term. It's hard to overstate the importance of the FTNS in shaping the direction of evolutionary theory. The FTNS came to typify the idea that evolutionary change is dominated by natural selection as a fitness optimizing force. Bill Hamilton appealed to this idea in his original paper on kin selection \cite{Hamilton:1964} and theorized that ``inclusive fitness'' is the fitness quantity that is maximized.

Subsequent work by population geneticists revealed however that mean fitness can decrease due to frequency-dependent selection \cite{Wright:1955,Lewontin:1970} or recombination among multiple genetic loci \cite{Kojima:Kelleher:1961,Moran:1964,Karlin:1975,Akin:1979}. Further, studies in the 1960s of the rates of molecular evolution \cite{Zuckerkandl:Pauling:1965,King:Jukes:1969} and levels of polymorphism \cite{Harris:1966,Lewontin:Hubby:1966} in a number of species spurred the development of the neutral \cite{Kimura:1968,Kimura:1983:book} and nearly neutral \cite{Ohta:1974,Ohta:1992} theories \cite{Ohta:Gillespie:1996} of molecular evolution. These theories posited that many mutations are weakly affected by natural selection, and thus their fate is governed mostly by genetic drift \cite{Kimura:1983:book}. Consequently, by the 1970s, evolutionary biologists were heavily debating both the relative role of natural selection versus other forces in shaping evolutionary patterns \cite{Gillespie:Langley:1974,Gillespie:1978}, and some biologists including Lewontin specifically questioned the importance of fitness maximization \cite{Karlin:1975,Gould:Lewontin:1979}. Maynard Smith was a clear proponent of fitness maximization and viewed it natural to ``[assume] that evolution has occurred by natural selection'' \cite[p. 31]{MaynardSmith:1978}. Thus, he proposed the ESS method developed by Price and himself as the appropriate tool for predicting phenotypic evolution and particularly for social traits. While Maynard Smith acknowledged that ESS models make a number of biological assumptions including that traits have a simple genetic basis (i.e., no dominance, epistasis, etc) and that appropriate genetic variation exits \cite{MaynardSmith:1978}, he believed that the limitations imposed by these assumptions are reasonable. This view that simplifying away genetic complexities or constraints is generally reasonable was termed the ``phenotypic gambit'' by Alan Grafen \cite{Grafen:1984} and came to dominate evolutionary theory in behavioral and evolutionary ecology \cite{Houston:McNamara:1999}. The near-singular focus of evolutionary game theory and the ESS method on natural selection has become less defensible given the recent and increasing abundance of genomic and transcriptomic data for a diverse set of species \cite[e.g.,][]{Kapheim:Pan:2015,Mikheyev:Linksvayer:2015,Warner:Mikheyev:2017,Kocher:Mallarino:2018,Warner:Mikheyev:2019}; specifically, some have argued that understanding these data requires moving beyond the gambit with more explicit consideration of complex genetic and demographic mechanisms \cite[e.g.,][]{Springer:Crespi:2011,Rittschof:Robinson:2014,Akcay:Linksvayer:2015,Cunningham:2020}. Question \ref{q:II} arises here and asks whether the ESS method can be modified to accommodate other evolutionary forces such as mutation and recombination. I will argue below that the ESS method is more general than originally imagined by Maynard Smith and Price \cite{Maynard-Smith:Price:1973} in that in can in fact be incorporated into an evolutionary framework where natural selection and other evolutionary forces combine to shape the short and long-term evolution of traits.

%%%
%%% Detailed outline for post intro
%%%

What is ESS? ESS is just phenotypes and fitness and an equilibrium and stability condition. This matches pop gen for simple genetics \cite{Eshel:1982}.


ESS really is a model of a ``long-term evolution'' process of invasion of mutant by resident where fitness determines invasion. This is the ``unbeatable strategy'' of Hamilton.

Long-term evolution depends on a fitness ``measure'' so ``fitness'' isn't just individual fitness. Really, we track the frequency or density of individual alleles (i.e., gene's eye view). This is lineage growth rate.

Lineage growth rate allows us to measure ESS in complex structure populations with kin and group selection. We measure the growth rate of the lineage in all the genetic and demographic environments it finds itself in.

Genetic environments can include other loci linked by recombination. This is the case of the modifier theory for recombination, mutation, migration, etc. Example from Liberman 2011.

Recognizing the importance of genetic parameters in ESS is the opposite of early work to justify ESS which tried to find conditions when ESS predicted same as pop gen but genetic parameters didn't appear.

One way forward to understanding the complex phenotypic evolution including social evolution should involve using ESS theory that incorporates more genetics including multiple loci. This is notably complementary to methods that study trait coevolution, e.g., Mullon and Lehmann.

%%%
%%% Fleshed out outline material
%%%

\section{What is an ESS?}

The definition of an ESS is relatively simple and only involves a measure of fitness, a set of phenotypes, and a stability condition. Let $\w(x,y)$ be the fitness that an individual with phenotype or strategy $x$ obtains when interacting with an individual with phenotype or strategy $y$. The phenotypes can be drawn from a set of discrete or continous set of values $X$ that represent a single trait like body size or propensity to cooperate. Fitness $\w$ for the moment is simply the survival rate (i.e., viability selection) but as we'll see we can define much more general fitness measures. The stability condition \cite{Maynard-Smith:Price:1973,Maynard-Smith:1974} says that a phenotype $\ess{x}$ is an ESS when no alternative strategy $y$ can improve upon the fitness that $\ess{x}$ receives when interacting with itself:
\begin{subequations}
  \label{eq:ess:def}
\begin{equation}
  \label{eq:ess:def:a}
  \w(\ess{x}, \ess{x}) \ge \w(y, \ess{x})
\end{equation}
for all phenotypes $y$. In the case that the strategy $y$ gets the same fitness interacting with $\ess{x}$ as $\ess{x}$ does with itself (equation \eqref{eq:ess:def:a} holds as an equality) then $\ess{x}$ must do better interacting with $y$ than $y$ does with itself:
\begin{equation}
  \label{eq:ess:def:b}
  \w(\ess{x}, y) > \w(y, y)
\end{equation}
\end{subequations}
The two conditions in \eqref{eq:ess:def} are equivalent to requiring that phenotype $\ess{x}$ receives higher average fitness than any alternative phenotype $y$ when $y$ is rare in the population \cite{Maynard-Smith:1974,Bishop:Cannings:1976}.

Notably, the conditions in \eqref{eq:ess:def} do not reference the genetic basis of the trait $X$ nor do they describe how the frequency or mean value of the trait changes over time. Population genetics models, on the other hand, are built to measure the dynamical process of evolution via the combined effects of natural selection and segregation, recombination, mutation, genetic drift, and other evolutionary forces \cite{Crow:Kimura:1970,Ewens:2004}. A natural question then is how the long-run behavior of a population genetic model (e.g., its equilibrium genotypes and induced phenotypes) compares to the equilibrium phenotypes obtained from an analogous ESS model. This question was tackled by mathematical population biologists beginning in the late 1970s and through 1980s and 1990s \cite[e..g,][]{Taylor:Jonker:1978,Hofbauer:Schuster:1979,Zeeman:1980,Eshel:1982,Hofbauer:Schuster:1982,Lessard:1984,Cressman:1988,Cressman:Hines:1984,Cressman:Hofbauer:1996,Hammerstein:1996,Weissing:1996,Eshel:1996,Eshel:Feldman:1984}. Early work in the simplest haploid asexual model with discrete strategies showed that the ESSs of obtained from condition \eqref{eq:ess:def} must be stable equilibria of haploid population genetic model \cite{Taylor:Jonker:1978,Hofbauer:Schuster:1979,Zeeman:1980}. This result isn't particularly surprising since natural selection is the only evolutionary force acting in the haploid asexual model. However, this idea was extended further by Ilan Eshel and Sabin Lessard




%%%
%%% old scraps and ramblings
%%%


\section{Stability in dynamical systems and the ESS}

For the behavior or evolutionary ecologist interested in a mathematical model of natural selection, an ESS model offers one main advantage over population genetic models: the ESS model can be formulated entirely in terms of phenotypes and ftiness and can avoid relying explicitly on genetic parameters (e.g., number of loci affecting the trait, recombination rates, etc) because the implicit genetic assumptions (i.e., additive interactions between alleles within a locus and across loci) imply a population genetic model with the same assumptions would generate the same ESSs \cite{Eshel:1982}.



\section{Long-term evolution}

Following the development of the ESS approach by Maynard Smith and Price \cite{Maynard-Smith:Price:1973}, mathematical biologists evaluated the approach in light of the older and established theory population genetics. Population genetic models can explicitly address how both natural selection and genetic mechanisms including mutation, segregation, and recombination affect allele and genotype frequency dynamics over time. These models typically assume that evolution changes the frequencies of a fixed set of genotypes and that the emergence of new genotypes through the large-effect mutations (i.e., the creation of new alleles and loci) occurs on a longer timescale than that of the model. For this reason, these population genetic models are often said to represent a ``short-term'' evolutionary process. Population genetic models showed that \cite{Cavalli-Sforza:Feldman:1978,Eshel:1982}



Long-term evolution cites \cite{Eshel:1996,Hammerstein:1996,Weissing:1996}






Though the idea of fitness optimization by itself has some intuitive appeal, high fitness individuals produce more surviving offspring who themselves are likely to be more fit than average, it needs an explicit evolutionary model in order to be evaluated mathematically and quantitatively.




\section{Acknowledgements}

\section{Funding}

\clearpage
%\setlength{\bibitemsep}{1pt}
\printbibliography

\end{document}

maybe my argument is then (see below) that population genetic model that include many forces can be converted into long-term evolution models.

So the idea should be something like ESS is really a mathematical formulation of the long-term evolution process which can include additional evolutionary forces other than natural selection.

The argument here is that invasion fitness / lineage fitness and its finite population analogue, fixation probability, are really just measures of evolutionary success that can include any potential evolutionary force. The maxima of these measures then yield the stable fixed equilibria of a long-term evolutionary process in a deterministic model or the modes of probability density in a stochastic model. Can appeal to physical notions of mean-field (?), potential wells, large deviation theory. Also appeal to stochastic stability and equilibrium selection from game theory (which borrows from physics). ESS then represents attracting points in phenotypic space and is informative even in the non-zero mutation and finite population size cases.
