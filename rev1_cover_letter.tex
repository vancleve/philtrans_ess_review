\documentclass[10pt,parskip=full,foldmarks=off,addrfield=off,backaddress=false,refline=dateleft,letterpaper]{scrlttr2}
\LoadLetterOption{kentucky}

\usepackage{xspace}

% Sender's information
\setkomavar{signature}{%
  \includegraphics[scale=0.3]{hand_sig.pdf}\\%
  Jeremy Van Cleve\\%
  Associate Professor\\%
  Department of Biology\\%
  University of Kentucky} % use PDF image of signature
\setkomavar{date}{\today}
\setkomavar{subject}{}

\usepackage[explicit]{titlesec}
\titleformat{\section}{\scshape \bfseries}{}{0ex}{#1}
\titlespacing{\section}{0pt}{1.0ex plus 0.25ex minus 0.25ex}{0.5ex plus 0.1ex minus 0.1ex}

\newenvironment{reviewerquote}{\begin{quote}\color{DarkBlue}\bfseries}{\end{quote}}
\newenvironment{response}{}{}

\begin{document}

% Recipient's information
\begin{letter}

  \opening{Dear Phil Trans B}

  Please find enclosed my revision of the manuscript entitled ``Evolutionarily stable strategy analysis and its links to demography and genetics through invasion fitness''. The Reviewers made a number of helpful and probing comments, and I have attempted to address each one thoroughly below. Reviewer comments are quoted in blue italics and my responses immediately follow. I have also uploaded a ``tracked changes'' version of the PDF to enable easy reference for changes that have been made in response to the reviews.

  \closing{Best regards,}

\section{Editor comments}

\begin{reviewerquote}
  Reviewer 1 has made several useful suggestions for clarity and they should be straightforward to implement in the revision.

  While Reviewer 2 also finds the manuscript well written, they are more critical and make several comments which should be addressed and may require more extensive revision.
\end{reviewerquote}

I would like to thank the Editor and Reviewers for their very detailed in insightful comments. I have copied each Reviewer's comments below followed by a detailed response with changes I've made in response to the comments.

\section{Referee 1 comments}

\begin{reviewerquote}
  This paper reviews the history and current state of the field of ESS analysis. The author links ESS analysis to the concept of invasion fitness, and discusses the application of invasion fitness to kin selection and evolution in variable environments Although I learned some valuable and interesting things, and I enjoyed reading the first half of this paper, I also found it difficult to understand some of the latter parts of the paper. Since reviews usually aim to clarify and give a (relatively) simple overview of the literature, I think the paper would benefit from some rewrites and from more careful notation and explanation of notation.
\end{reviewerquote}

Thank you for the excellent feedback. I've endeavored to improve the latter parts of the paper and provide clearer explanations. Please see my responses below to the reviewer's specific comments.

\begin{reviewerquote}
Confusion/clarification/notational questions:\\
- You are using $x$ and $y$ to denote both the phenotypes, and their frequencies, and I’m not always sure which one you mean. E.g. up until line 138, x and y have been defined as denoting the phenotypes, so I assume that $w(x,y)$ in line 127 means that $x$ and $y$ are interacting, and it does not mean that w is a function of the frequencies of $x$ and $y$? It would help me a lot if you used different symbols for these two things, e.g. maybe $p_x$ and $p_y$ for the frequencies?\\
- Line 135: So now we have x denoting both the phenotype and the frequency of the phenotype, and then we have $x_{ij}$ denoting mating preference of genotype $A_i$ $A_j$. Is it possible to use a different symbol for $x_{ij}$?
\end{reviewerquote}

Thanks for mentioning this. I did start the section talking about two strategies, $x$ and $y$, and thus do mean as suggested above that $w(x,y)$  is a function of phenotypic values $x$ and $y$ and not of frequencies of those strategies in the population. However, I think what may be causing confusion is that then I moved to allowing individuals to choose strategies $x$ and $y$ probabilistically, so now mixed strategies are allowed and the strategy is technically the probability of selecting strategy $x$ (with complementary probability of selecting strategy $y$). Since the probabilities $x_{ij}$ are strategies, I'd like to keep the same notation $x$ for the probabilities.

However, a reworking of this section and some notation change I hope will make things simpler and clearer. I've started off assuming in section 2 that the strategies are probabilities of selecting behavior $C$ when individuals have two possible behaviors, $C$ and $D$. Now, the equation on line 138 (now eqn 2) is more clearly described as the mean strategy in the population or mean probability of selecting behavior $C$ and is no longer confusingly described as a frequency. I've added additional explanation in sections 2 and 3 to make the description of the phenotypes clearer.

\begin{reviewerquote}
- I’m not sure I understand what “the ESSs obtained by applying condition (1) to the payoff matrix $A$” means. Does it mean that we are considering some trait to evolve until the entries of the pay off matrix satisfies the ESS conditions, are x and y the evolving phenotypic trait? Does it means that at some frequencies of x and y, the entries of the pay off matrix satisfy the ESS condition (related to the previous question)?
\end{reviewerquote}

Yes, I agree that this is unclear. I have rewritten this paragraph to better describe both the population genetic model and the ESS analysis. The ESS analysis is simply to use the condition (1) to determine which of the two strategies is ESS or if there is a mixed strategy ESS. It should be clear now that the ESS value of the phenotype is the same value of the genotype frequencies that are stable in the population genetic model.

\begin{reviewerquote}
  - Line 184: Equation for mean fitness: what happened to $y$? Shouldn’t the mean fitness in the population also be a function of phenotype $y$?
\end{reviewerquote}

The equation is correct as written and the confusion here stems from the earlier issue the reviewer highlighted. In this scenario, the phenotype $x$ is the probability of selecting behavior $C$ (which I was calling ``$x$'' before confusingly) and thus the probability of selecting behavior $D$ is simply $1-x$. Thus, only one phenotypic variable, $x$, is needed to calculate the mean fitness. I have added additional clarification around this equation in addition to the earlier text clarifying the phenotypes.

\begin{reviewerquote}
  - Line 182: “the fitness function is linear”, linear in what/a linear function of what?
\end{reviewerquote}

I've had added that the fitness $w(x,y)$ is a linear function of its arguments, which are the phenotypes of the two interacting individuals. Linearity here is reasonable given the phenotype is a probability of selecting among two pure strategies.

\begin{reviewerquote}
  - Equation 3: $u_j$ is not yet defined I think? $u_{xy}$ has been defined as an entry of the pay off matrix, but this $u$ only comes with one index?
\end{reviewerquote}

Ah, apologies. The $u_{j}$ are elements of the eigenvector mentioned above equation 2 and I now spell that out more clearly. I have also changed notation for the payoff matrix to $W$ to make things clearer.

\begin{reviewerquote}
  - Line 212-213: “under the right assumptions”, and earlier in that paragraph “given the general genetic and demographic assumptions above”. I read through the above again to try and find what these assumptions are but had trouble locating them, except for maybe “the fitness function is linear” and “mutant allele is rare”? Can you repeat/list the assumptions necessary for Result 1 to hold?
\end{reviewerquote}

I have reworked those sentences and have added a list of the assumptions after Result 1 with some discussion of how they might be relaxed.

\begin{reviewerquote}
  - I guess $u_k(y,x)$ on line 305 is an element in the vector $u(y,x)$ from line 292?
\end{reviewerquote}

Yes, I now specify that in the text.

\begin{reviewerquote}
  - What are the dimensions of the matrix $A(y,x)$ in line 287? I’ve never encountered vectors used as matrix indices before (as in $a_{k’k}$, usually matrix indices are scalars), and I’m not sure how to interpret the notation? If $A$ contains all possible $k$ and $k’$ vectors, its dimensions must quickly get very large?
\end{reviewerquote}

Yes, the dimensions of the $\mathbf{A}$ matrix are all possible $\vec{k}$ and $\vec{k}$ vectors and the size quickly gets very. It may not be the most computationally efficient formalism but works well enough as a conceptual approach. The notation works by assuming that one creates an ordering for the possible states in $\vec{k}$ and then indexes the matrices and vectors by that ordering. I now specify this in the third paragraph of section 6.

\begin{reviewerquote}
  - Really enjoyed the section on variable environments.
\end{reviewerquote}

Excellent, I'm glad!

\begin{reviewerquote}
Minor comments:\\
- Line 53: I’m not sure what it means for a method to be orthogonal to issues
\end{reviewerquote}

Corrected.

\begin{reviewerquote}
Typos:\\
- Line 87: exits -> exists
- Line 97: in that in can in\\
- Line 155 an genetic -> a genetic\\
- Line 347 “One of the important features of Result 1 is that shows how”
\end{reviewerquote}

Fixed.

\section{Referee 2 comments}

\begin{reviewerquote}
This manuscript attempts to review the 50-year history of the ESS concept. In view of the more than 50,000 “ESS articles” published since 1973, this is a highly ambitious task. Although I agree with some of the take-home messages of the manuscript, I am not entirely convinced that Van Cleve has succeeded in his mission.
\end{reviewerquote}

Thanks for the critical comments! I think indeed the task is a big one but I hope the revisions have improved the manuscript. The review certainly can't cover the whole history but I hope its captured a reasonable slice.

\begin{reviewerquote}
From the start, the ESS history was multi-faceted. The 1973 paper of Maynard Smith and Price was based on an earlier manuscript of Price (which I could inspect when visiting Maynard Smith many years ago). Here, the main novel idea was not the ESS concept but rather “strategy thinking”, adopted from game theory (strategy = recipe for condition-dependent behaviour). Price argued that a “mouse” strategy (nowadays called Dove) could be rapidly outcompeted by a more aggressive strategy, but that this would not be the case for a ”retaliator” strategy, which, in the absence of aggressive individuals, leads to precisely the same behaviour as the “mouse” strategy. From this, Price drew the (for him) most important conclusion that the “organisation of behaviour” (that is, the “strategy” underlying behaviour) is crucial for understanding behavioural evolution. In contrast, Maynard Smith had one main goal: to publish his ESS concept before Hamilton could work out his ideas on “unbeatable strategies” in more detail.

I mention all this because Van Cleve starts talking about evolutionary “game theory”, while game aspects (like strategy thinking) play a minor role in his review. He is not really interested in behavioural interactions, but in generalizing the ESS concept to more complicated settings. At first, I found this quite confusing, as, from the historical perspective, it is crucial that the ESS concept was first developed in a game context. In fact, Maynard Smith’s derivation of the ESS conditions (1a) and (1b) only makes sense for two-person matrix games, as in this case the fitness function w(x,y) is linear in both components.
\end{reviewerquote}

There is some very interesting history here that I was not aware of. The reviewer has a unique perspective that I hope they write up soon!

With respect to the two-person matrix game aspect, I now make the introduction of conditions (1a) and (1b) refer specifically to a pairwise game with to discrete strategies where the phenotype is the probability of selection among the discrete strategies (i.e., a a mixed strategy). I also introduce both the prisoner's dilemma and the snowdrift/hawk-dove game as examples to give the reader more context going into the more theoretical sections.

I realize I don't focus on the ``strategy thinking'' aspect too much, though there is a little more of that aspect with the prisoner's dilemma and hawk-dove examples. I am writing from my background in population genetics where I learned about ESS as is related to general questions related to the dynamics of genetic and phenotypic evolution. As the reviewer says, there are many facets to the development of ESS, theory and I think this review is aiming for the evolutionary dynamic aspects and leaving the strategic thinking perspective to other efforts. I think this isn't an unprecedented perspective either. For example, Hofbauer and Sigmund in their 1998 book frame ESS in general around a population dynamic perspective.

\begin{reviewerquote}
  Various authors later “corrected” this “mistake” and derived more general ESS concepts for non-linear fitness functions. It is a pity that all this is not mentioned by Van Cleve...Anyway, it is now quite obvious that Maynard Smith would have designed the ESS conditions quite differently if he had considered the possibility of a non-linear fitness function. It soon became clear that such functions can have quite different evolutionary properties, like “Garden of Eden” scenarios where an ESS is not attainable, or “evolutionary branching” scenarios where the population converges to an evolutionarily unstable strategy. Later work on this resulted in Adaptive Dynamics theory, which is not mentioned at all in Van Cleve’s manuscript, although the invasion fitness concept of Adaptive Dynamics is more sophisticated than the variant propagated by Van Cleve. Interestingly, from the Adaptive Dynamics perspective, the second ESS condition of Maynard Smith is not a condition for “evolutionary stability” at all, but rather a condition for “convergence stability”. This has been noticed before (in writing) by various colleagues, including Eshel, Lessard, Metz, and Dieckmann. It would have been very interesting to hear Van Cleve’s position regarding this reinterpretation of “ESS history”.
\end{reviewerquote}

Yes, I have largely left out reviewing adaptive dynamics, which is not directly intentional per se but more because I wanted to build around the ESS condition in the original 1973 paper as the Phil Trans issue commemorates 50 years since that paper. However, adaptive dynamics is of course extremely important to evolutionary game theory and to the concept of invasion fitness. I've added XXX...
%%% Add something about convergence stability to discussion.

As to whether invasion fitness in adaptive dynamics is more general than what I discuss here, I would guess the reviewer is thinking of cases where short-term evolution reaches a non-equilibrium attractor like a cycle or chaotic attractor and invasion fitness must then really be measured in the sense of a Lyapunov exponent. I agree those cases are important but I do not mention them so as to keep the scope of the review from growing too much. It should be noted too that the assumptions and tools employed in typical deterministic and stochastic adaptive dynamics models ignore the non-equilibrium cases too as they are difficult to handle in general. One of those assumptions is often, as I discuss in reference to long-term evolution, infrequent mutation, which when combined with finite population size, leads to at most two genotypes existing at a time (see the Champagnat 2006 ref) and one of them either going extinct or fixing before the arrival of the next mutation.

Finally, I totally agree that the second ESS condition from the 1973 paper is really the convergence stability condition. My review wasn't structured to synthesize the original ESS conditions with the stability and convergence conditions from adaptive dynamics but this would be the basis from a great review or synthesis article. %% Add something here maybe if I write convergence stability paragraph?

\begin{reviewerquote}
  Indeed, non-linearities arise, for example, in “games between relatives”. How to deal with this (and, hence, how to capture kin selection in the ESS formalism) led to a vivid exchange of arguments between Grafen and Maynard Smith in the late 1970s and the early 1980s. Grafen pointed out that his approach (corresponding to an “actor-modulated approach”) leads to quite different outcomes than Maynard Smith’s inclusive fitness approach. As kin selection is a central theme of Van Cleve’s manuscript, I would have expected to hear about these early debates.
\end{reviewerquote}

I think the reviewer may be referring to Maynard Smith (1978, Ann Rev Ecol Syst), Grafen (1979, Anim Behav), and Hines and Maynard Smith (1979, JTB).

\begin{reviewerquote}
  The middle part of Van Cleve’s review (Sections 3 and 4) contains some interesting observations, but the results are often stated in a relatively sloppy manner (i.e., without mentioning the conditions under which they hold).
\end{reviewerquote}

I agree that I need to be clearer about the conditions necessary the results in sections 3 and 4 to hold. I've now added a bulleted list at the end of section 4 that lists all of the conditions and also discuss how restrictive they may or may not be.

\begin{reviewerquote}
  In fact, the references referred to in the review all make quite limiting assumptions, such as a linear fitness function (in both components) or a system with only two phenotypes. Take, for example, the discussion of internal versus internal stability in Section 4, two very interesting concepts indeed. These concepts are mainly useful in systems with many different strategies. In contrast, the central Result 1 of that Section only applies to systems with two pure strategies (or phenotypes), and it does not hold anymore in three-strategy games like Rock-Scissors-Paper (this is, for example, explicitly mentioned in the key reference 103). Obviously, this limitation of Result 1 should have been mentioned.
\end{reviewerquote}

Yes, this limitation is important. I now specify it clearly as the first assumption in the bulleted list at the end of section 4.

\begin{reviewerquote}
  In fact, various counter-examples to both parts of Result 1 were constructed in the evolutionary branch of economic game theory (most notably by Friedman and Nachbar). Like my colleagues from economics, I also spent several years on attempts to extend Result 1 to more than two phenotypes. Like them, I had to conclude that long-term evolution does not “save” the ESS concept when more than two strategies are around.
\end{reviewerquote}

%% mention this more clearly at the end of section 4 and/or in the discussion


\begin{reviewerquote}
  But also the case of only two phenotypes is less clear than presented by Van Cleve. A good example can be found in the excellent book of Karlin and Lessard on sex ratio evolution (which, historically, was very important for the idea of long-term evolution). Karlin and Lessard realized that genetic constraints often prevent the evolution of an “evolutionarily stable” sex ratio (sensu Maynard Smith). Around this time, they collaborated with Eshel, and together the three considered long-term evolution. Karlin and Lessard decided to conduct a long-term simulation study by once in a while adding new genetic variants, in order to remove genetic constraints. This study resulted, however, in deception, as it did not converge at all. Sometimes the trajectory converged to a state close to the ESS sex ratio, but consecutively, it always moved away again. Later, I found the same in my own simulations, both in the context of sex ratio evolution and in the context of two-strategy games. Accordingly, I am much less convinced than Van Cleve that “long-term evolution” has been shown to “save” the ESS concept.
\end{reviewerquote}

%% ugh. ok take care of this

\begin{reviewerquote}
  In contrast to the earlier sections, which are broadly accessible, Sections 6 and 7 are technically more sophisticated and will most likely only be understandable to experts. This holds, in particular, for Section 7. I wonder whether it is really useful to present Fig. 1 without discussing the underlying assumptions in sufficient depth and detail.
\end{reviewerquote}

I agree that section 6 is more technical. I have added XXX..
%% add here

I have tried to keep Section 7 to a minimum of technical depth. I do not agree that its understandable to only technically sophisticated readers. However, it is relevant to add some of the assumptions for the modifier models discussed and I now do that in XXXX
%% add here

\begin{reviewerquote}
  The “reduction principle” is a wonderful tool but I would have liked to see a more critical discussion of its applicability.
\end{reviewerquote}

I mention the reduction principle since it uses external stability to show the general evolutionary principle that selection in a constant generates purifying selection for the highest fitness genotype and against genetic variation created by other genotypes. Varying environments don't exhibit the reduction principle since genetic or phenotypic variation can be adaptive in these cases. I don't see how discussing technical criticisms of the reduction principle advances the purpose of the section, which is to show how the ESS concept as it relates to external stability can study the evolution of genetic and demographic parameters in the case of varying environments.

\begin{reviewerquote}
Generally, the manuscript is well-written. But as mentioned above, there is a certain imbalance in this manuscript: no (restrictive) assumptions are stated in the first sections (although the experts would have liked to see them spelt out explicitly), while the treatment in the later sections is only understandable to experts in the field.
\end{reviewerquote}

I hope I have addressed these criticisms as described above and hope these changes improves the manuscript!

\end{letter}

\end{document}
